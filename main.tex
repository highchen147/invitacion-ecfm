\documentclass[letterpaper,12pt]{article}
\usepackage[spanish]{babel}
\usepackage[utf8]{inputenc}
\usepackage{graphicx}
\usepackage{background}
\usepackage{geometry}
\usepackage{eso-pic}
\usepackage{tikz}

% Ajusta los márgenes según tus preferencias
\geometry{
paperheight=25cm,
paperwidth=21cm, 
top=2cm, bottom=2cm, left=2.5cm, right=2.5cm}

% Ecuación mamona de fondo a elección
\backgroundsetup{
  scale=4,
  angle=0,
  color=black,
  opacity=0.12,
  position=current page.south,
  hshift = -0.5cm,
  vshift=0.7cm,
  contents={%
\(\displaystyle \rho_{t} + (\rho v)_{x} = 0\)\hspace{6mm}
  }
}


\begin{document}
\pagestyle{empty}


% Margen alrededor de la hoja. Si cambian los márgenes del documento se desajusta. TODO: automatizar el margen.
\AddToShipoutPicture{%
  \begin{tikzpicture}[remember picture, overlay]
    \draw[line width=1pt] (1cm,1cm) rectangle (20cm,24cm);
  \end{tikzpicture}
}

\begin{center}
	% Logo escuela/facultad
	\includegraphics[width=0.55\textwidth]{1.png}\hfill
	% Logo USAC/otra U
    \includegraphics[width=0.28\textwidth]{usaclogo.png}

    \vspace{1cm}

    {\large El estudiante}

    \vspace{1cm}
	% NOMBRE DEL GRADUANDO
    \textbf{\Large Rodrigo Rafael Castillo Chong}

    \vspace{1cm}
\large
    {tiene el honor de invitarle al solemne acto de graduación y examen público previo a serle conferido el título de}

    \vspace{1cm}
	% TÍTULO UNIVERSITARIO A ADQUIRIR
    \textbf{\Large Licenciado en Física Aplicada}

    \vspace{1cm}
	% LOCALIDAD O DATOS ADICIONALES
    {\large que tendrá lugar en el \textbf{Auditorio Francisco Vela} de la facultad de Ingeniería, en la }

    \vspace{1cm}
    % NOMBRE UNIVERSIDAD
    {\Large \textbf{Universidad de San Carlos de Guatemala}}
    \vspace{1cm}
	% FECHA Y HORA
    \textbf{\large [Miércoles 13 de marzo] \hspace{0.5cm} [3:00pm]}

    \vspace{1cm}
	% RECEPCIÓN U OTROS DATOS
    \textbf{\large [Recepción: Vesuvio Majadas, 4:30pm]}

\end{center}

\end{document}
